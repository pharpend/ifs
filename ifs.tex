\documentclass[12pt,letterpaper,oneside]{memoir}

\usepackage{amsmath}
\usepackage{amsfonts}
\usepackage{amssymb}
\usepackage{amsthm}
\usepackage[date=iso8601,urldate=iso8601]{biblatex}
\usepackage{centernot}
\usepackage{color}
\usepackage{datetime2}
\usepackage{listings}
\usepackage{mathtools}
\usepackage{tabu}
\usepackage{url}

\usepackage[scaled=0.85]{FiraMono}
\usepackage[T1]{fontenc}

\usepackage[hidelinks]{hyperref}
\usepackage{cleveref}

\definecolor{mygreen}{rgb}{0.3,0.6,0.3}
\definecolor{mygray}{rgb}{0.8,0.8,0.8}
\definecolor{mymauve}{rgb}{0.58,0,0.82}
\lstset{ %
  % backgroundcolor=\color{white},   % choose the background color; you must add \usepackage{color} or \usepackage{xcolor}
  basicstyle=\ttfamily,     % the size of the fonts that are used for the code
  breakatwhitespace=false,         % sets if automatic breaks should only happen at whitespace
  breaklines=true,                 % sets automatic line breaking
  captionpos=\null,                    % sets the caption-position to bottom
  commentstyle=\color{mygreen},    % comment style
  deletekeywords={...},            % if you want to delete keywords from the given language
  escapeinside={\%*}{*)},          % if you want to add LaTeX within your code
  extendedchars=true,              % lets you use non-ASCII characters; for 8-bits encodings only, does not work with UTF-8
  frame=single,                    % adds a frame around the code
  keepspaces=true,                 % keeps spaces in text, useful for keeping indentation of code (possibly needs columns=flexible)
  keywordstyle=\bfseries,       % keyword style
  % language=\null,                % the language of the code
  % morekeywords={*,...},            % if you want to add more keywords to the set
  numbers=left,                    % where to put the line-numbers; possible values are (none, left, right)
  numbersep=5pt,                   % how far the line-numbers are from the code
  numberstyle=\tiny\ttfamily,    % the style that is used for the line-numbers
  postbreak=\raisebox{0ex}[0ex][0ex]{\ensuremath{\color{red}\hookrightarrow\space}},
  rulecolor=\color{mygray},        % if not set, the frame-color may be changed on line-breaks within not-black text (e.g. comments (green here))
  showspaces=false,                % show spaces everywhere adding particular underscores; it overrides 'showstringspaces'
  showstringspaces=false,          % underline spaces within strings only
  showtabs=false,                  % show tabs within strings adding particular underscores
  stepnumber=1,                    % the step between two line-numbers. If it's 1, each line will be numbered
  stringstyle=\color{mymauve},     % string literal style
  tabsize=2,                       % sets default tabsize to 2 spaces
  title=\lstname,                   % show the filename of files included with \lstinputlisting; also try caption instead of title
  caption=\lstname ,                  % show the filename of files included with \lstinputlisting; also try caption instead of title
}

\addbibresource{ifs.bib}
\nocite{*}

\theoremstyle{plain}
\newtheorem{definition}{Definition}[section]

\theoremstyle{definition}
\newtheorem{remark}[definition]{Remark}

\newcommand{\psubof}{\subsetneqq}
\newcommand{\subof}{\subseteq}
\newcommand{\parens}[1]{\left( #1 \right)}
\newcommand{\brackets}[1]{\left[ #1 \right]}
\newcommand{\braces}[1]{\left\{ #1 \right\}}
\newcommand{\abs}[1]{\left| #1 \right|}
\newcommand{\norm}[1]{\left\| #1 \right\|}
\newcommand{\mset}[1]{\braces{\, #1 \,}}
\newcommand{\scomp}[2]{\mset{#1 \::\: #2}}
\newcommand{\floor}[1]{\left\lfloor #1 \right\rfloor}
\newcommand{\Z}{\mathbb{Z}}
\newcommand{\N}{\mathbb{N}}
\newcommand{\Q}{\mathbb{Q}}
\newcommand{\R}{\mathbb{R}}
\newcommand{\C}{\mathbb{C}}
\newcommand{\ctext}[1]{$$\text{#1}$$}
\newcommand{\term}{\textbf}
\newcommand{\divides}{\mid}
\newcommand{\code}{\lstinline}
\newcommand{\lstin}[1]{\lstinputlisting{listings/#1}}
\newcommand{\dolsign}{\texttt{\$}}

\begin{document}
\title{Idris from Scratch}
\author{Peter Harpending \texttt{<peter.harpending@utah.edu>}}
\maketitle

\tableofcontents

\chapter{Getting set up \& toying around in the REPL}

Idris is a dependently typed, purely functional language. We'll learn
what all of those things mean in a second. For now, it's worth noting
that Idris guarantees a degree of code quality that you simply can't
find in most other programming languages.

\section{Setup}

The Idris compiler is written in Haskell, so you need to install
Haskell first. The Haskell download webpage
(\url{https://www.haskell.org/downloads}) documents several methods of
installing Haskell. My personal recommendation is to use Stack
(\url{http://haskellstack.org}). It makes whole sets of problems with
Haskell packaging go away.

The instructions for installing Stack change somewhat
frequently. There are also too many operating systems for me to fit in
this text. Therefore, I will direct you to the Stack documentation
(\url{https://docs.haskellstack.org/en/stable/README/#how-to-install}). Find
your operating system, and follow the instructions.

If you can't find your operating system, I would suggest opening a bug
in Stack's bug tracker
(\url{https://github.com/commercialhaskell/stack/issues}), and ask for
help. In my experience, the Stack people have been extremely helpful
in resolving my issues.

Moving on. It's worth noting that I own several Linux machines, but I
do not own a Macintosh or a Windows machine. Macintosh is similar
enough to Linux that most of my instructions should still work. I have
positively no idea how to do any of this set up on Windows. If someone
does, please do add instructions to this text. The git repository is
located at \url{https://github.com/pharpend/ifs}.

Once you have stack installed on your system, you'll need
to run some more commands to set things up

\lstin{stack-setup.txt}

The \texttt{\$} indicates that you're supposed to open a terminal, and
run whatever command comes after it. The commands I gave you will
update Stack's list of available packages, and then install the
Haskell compiler, ghc.

To install Idris, run \lstin{stack-install-idris.txt} in a terminal.

If you're on Unix, and it gives you an error about a missing library
dependency, try installing a package by that name, and suffixing
\code{-dev}, \code{-devel}, or some such. For instance, I was missing
the \code{ncurses} library. I installed the package
\code{ncurses-devel}, and that seemed to fix my problem. (I'm on
Fedora 24, currently).


\printbibliography
\end{document}

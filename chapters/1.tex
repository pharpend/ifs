\chapter{Getting set up \& toying around in the REPL}

Idris is a dependently typed, purely functional language. We'll learn
what all of those things mean in a second. For now, it's worth noting
that Idris guarantees a degree of code quality that you simply can't
find in most other programming languages.

\section{Setup}

The Idris compiler is written in Haskell, so you need to install
Haskell first. The Haskell download webpage
(\url{https://www.haskell.org/downloads}) documents several methods of
installing Haskell. My personal recommendation is to use Stack
(\url{http://haskellstack.org}). It makes whole sets of problems with
Haskell packaging go away.

The instructions for installing Stack change somewhat
frequently. There are also too many operating systems for me to fit in
this text. Therefore, I will direct you to the Stack documentation
(\url{https://docs.haskellstack.org/en/stable/README/#how-to-install}). Find
your operating system, and follow the instructions.

If you can't find your operating system, I would suggest opening a bug
in Stack's bug tracker
(\url{https://github.com/commercialhaskell/stack/issues}), and ask for
help. In my experience, the Stack people have been extremely helpful
in resolving my issues.

Moving on. It's worth noting that I own several Linux machines, but I
do not own a Macintosh or a Windows machine. Macintosh is similar
enough to Linux that most of my instructions should still work. I have
positively no idea how to do any of this set up on Windows. If someone
does, please do add instructions to this text. The git repository is
located at \url{https://github.com/pharpend/ifs}.

Once you have stack installed on your system, you'll need
to run some more commands to set things up

\lstin{stack-setup.txt}

The \texttt{\$} indicates that you're supposed to open a terminal, and
run whatever command comes after it. The commands I gave you will
update Stack's list of available packages, and then install the
Haskell compiler, ghc.

To install Idris, run \lstin{stack-install-idris.txt} in a terminal.

If you're on Unix, and it gives you an error about a missing library
dependency, try installing a package by that name, and suffixing
\code{-dev}, \code{-devel}, or some such. For instance, I was missing
the \code{ncurses} library. I installed the package
\code{ncurses-devel}, and that seemed to fix my problem. (I'm on
Fedora 24, currently).
